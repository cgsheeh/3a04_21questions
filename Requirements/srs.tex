\documentclass[titlepage]{article}

% Imported Packages
%------------------------------------------------------------------------------
\usepackage{amssymb}
\usepackage{amstext}
\usepackage{amsthm}
\usepackage{amsmath}
\usepackage{enumerate}
\usepackage{fancyhdr}
\usepackage[margin=1in]{geometry}
\usepackage{graphicx}
\usepackage{extarrows}
\usepackage{setspace}
\usepackage{longtable}
%------------------------------------------------------------------------------

% Header and Footer
%------------------------------------------------------------------------------
\pagestyle{fancy}  
\lhead{Group 2}
\chead{SFWR 3A04}
\rhead{Requirements Specification}
\renewcommand\headrulewidth{0.4pt}                                      
\renewcommand\footrulewidth{0.4pt}                                    
%------------------------------------------------------------------------------

% Title Details
%------------------------------------------------------------------------------
\title{\textbf{21 Questions - Requirements Specification}}
\author{Gabriel Lopez de Leon - INSERT MACID - 1310514\\Curtis Milo - milocj - 1305877\\Max Moore - moorem8 - 1320009\\Alex Rahman - rahmaa25 - 1305735\\Connor Sheehan - sheehacg - 1330964}


\date{February 8th, 2016}                            
%------------------------------------------------------------------------------

% Document
%------------------------------------------------------------------------------
\begin{document}
	
	\maketitle
	\vspace{4cm}	
	\abstract{\begin{center}
			Requirements Specification for the 21 Questions phenomenon identifier project in SFWR ENG 3A04 at McMaster University.
		\end{center}}
		\newpage
		\tableofcontents
		\newpage
		\section{Introduction}
		\label{sec:introduction}
		% Begin Section
		
		This section of the SRS should provide an overview of the entire SRS.
		
		\subsection{Purpose}
		\label{sub:purpose}
		% Begin SubSection
		\begin{enumerate}[a)]
			\item Delineate the purpose of the SRS
			\item Specify the intended audience for the SRS
		\end{enumerate}
		% End SubSection
		
		\subsection{Scope}
		\label{sub:scope}
		% Begin SubSection
		\begin{enumerate}[a)]
			\item Identify the software product(s) to be produced by name (e.g., Host DBMS, Report Generator, etc.)
			\item Explain what the software product(s) will, and, if necessary, will not do
			\item Describe the application of the software being specified, including relevant benefits, objectives, and goals
			\item Be consistent with similar statements in higher-level specifications (e.g., the system requirements specification), if they exist
		\end{enumerate}
		% End SubSection
		
		\subsection{Definitions, Acronyms, and Abbreviations}
		\label{sub:definitions_acronyms_and_abbreviations}
		% Begin SubSection
		\begin{itemize}
			\item \textbf{Area of Interest} -
			The phenomenon initially unknown to the system but known to the user. 		Throughout this document it will be referred to as the \textit{\textbf{AOI}}.
			\item \textbf{API} - Short form for application programming interface.
			\item \textbf{IT} - Short form for information technology.
			\item \textbf{SRS} - Short form for Software Requirements Specification, refers to this document.
			\item \textbf{Query} - A set of carefully selected words used in order to describe or pin point a location on a map.
			\item \textbf{Questions} - The method by which each expert receives information for analysis. These \textit{questions} will be answered primarily with yes or no answers.
			
		\end{itemize}
		% End SubSection
		
		\subsection{References}
		\label{sub:references}
		% Begin SubSection
		\begin{enumerate}[a)]
			\item Provide a complete list of all documents referenced elsewhere in the SRS.
			\item Identify each document by title, report number (if applicable), date, and publishing organization.
			\item Specify the sources from which the references can be obtained.
		\end{enumerate}
		% End SubSection
		
		\subsection{Overview}
		\label{sub:overview}
		% Begin SubSection
		\begin{enumerate}[a)]
			\item Describe what the rest of the SRS contains.
			\item Explain how the SRS is organized.
		\end{enumerate}
		% End SubSection
		
		% End Section
		
		
		\section{Overall Description}
		\label{sec:overall_description}
		% Begin Section
		
		\subsection{Product Perspective}
		\label{sub:product_perspective}
		% Begin SubSection
			\indent \indent 21 Questions is simple location identification application which asks the user a total of 21 questions to try and identify the \textbf{AOI}. 21 Questions is very similar to other identification applications such as \textit{Akinator} or the game \textit{20Q}. \textit{Akinator} is a web based program where the user thinks of a character and answers a series of questions, leading to the program guessing the character the user has thought of. \textit{20Q} or 20 Questions is the original inspiration for our program, like \textit{Akinator}, the user thinks of something but in this case it can be any person, place or thing. The game would then ask the user a total of 20 questions to try and guess what the user is thinking of. To differentiate our application with \textit{20Q}, 21 Questions focuses on locations only with an end goal of outputting the result through \textit{Google Maps}. Overall, the system is independent and only uses the Google API as its only resource outside of the system.
		% End SubSection
		
		\subsection{Product Functions}
		\label{sub:product_functions}
		% Begin SubSection
		The following is a list of product functions that the software will perform:
		\begin{enumerate}[a)]
			\item The system identifies an AOI using three different experts.
			\begin{itemize}
				\item This is one of the main requirements and functionality of 21 Questions, the system uses three different experts which are independent of each other to reach a resulting output to user inputs.
			\end{itemize}
			\item The system forms a query using the answers inputted by the user and is then outputted through Google Maps.
			\begin{itemize}
				\item From the users' answers to the preset questions, \textbf{queries} are formed and used to find the \textbf{AOI}. A set of queries are stored in a database which is referenced and used along with the Google maps API to display the resulting location on a map.
			\end{itemize}
			\item The system allows for users to add new queries.
			\begin{itemize}
				\item The ability to add new queries for new locations is an important functionality of the application as it allows the database to continuously grow.
			\end{itemize}
			\item The system shall allow for voice commands as a possible input to answer the questions.
			\begin{itemize}
				\item The use of voice commands will be our innovative feature and will allow for a wider range of users to be able to use the app. Adding the use of voice commands as an input improves the applications accessibility and allows for those with visual impairment, such as blind persons to be able to use 21 Questions.
			\end{itemize} 
		\end{enumerate}
		% End SubSection
		
		\subsection{User Characteristics}
		\label{sub:user_characteristics}
		% Begin SubSection
		\begin{enumerate}[a)]
			\item General Characteristics of Intended Users 
			\begin{itemize}
				\item 21 Questions is a location identifier application whose intended use is for any user above the age of ten. The application requires minimal training, experience or technical expertise to use, and can be easily picked up and used by anyone. 21 Questions can be used as a game where users select a location and see if the software can identify it based on the user's answers to a set of predefined questions. The application can also be used to identify a location, structure, or establishment as answering the questions will output the result via Google maps. 
			\end{itemize}
			\item Reasoning for Requirements
			\begin{itemize}
				\item As later stated in the non-functional requirements, any user can easily learn how to use the application. With game like aspects, 21 questions is meant for users of all ages and thus it should be simple enough that users of a younger audience will be able to play while users of an older audience will still find the game appealing.
				\item The application targets a wide variety of users with different ethnic backgrounds and thus must be respectful of everyones cultures, religious belief and political preferences.
			\end{itemize}
		\end{enumerate}
		% End SubSection
		
				
		\subsection{Constraints}
		\label{sub:constraints}
		% Begin SubSection
		The following constraints are to limit the design of the software to be:
		
		\begin{enumerate}[a)]
			\item The system must have three experts modules in which will determine an aspect about the location to identify.
			\item The systems software experts are not able to communicate between each other.
			\item The system should use the Google Maps API.
			\item The should must be designed so that one could swap out any expert for another.
			\item The system must have access to the Internet.
			\item The system must encrypt all transmitted messages.
			\item The system must abide to all the Canadian laws.
			\item The system must abide to all terms and conditions set by the Google Maps API.
			
		\end{enumerate}
		% End SubSection
		
		\subsection{Assumptions and Dependencies}
		\label{sub:assumptions_and_dependencies}
		% Begin SubSection
		\begin{enumerate}[a)]
			\item Assumptions
			\begin{itemize}
				\item The system can only find locations within the set domain.
				\item The user will only use the system to identify real locations.
				
			\end{itemize}
			\item Dependencies
			
			\begin{itemize}
				\item The system is dependent on the functionality the Google Maps API.
				\item The system's performance is dependent on the speed and quality of the user's Internet.
				\item The system's performance is dependent on the operating systems overhead.
				
			\end{itemize}
		\end{enumerate}
		% End SubSection
		
		\subsection{Apportioning of Requirements}
		\label{sub:apportioning_of_requirements}
		% Begin SubSection
		\begin{enumerate}[a)]
			\item Letting the user add locations to the existing set of locations.
			\item Using social media to allow the user to share the location that the system predicted.
			\item Suggest locations based off of interest and large news stories through social media.
			
		\end{enumerate}
		% End Section
		
		\section{Functional Requirements}
		\label{sec:functional_requirements}
		% Begin Section
		
		The system will react to a set of key stimuli based on the interests of a set of people. The system will provide output based on the following requirements:
		
		\begin{enumerate}[{BE1}]
			
			% Business Event 1
			\item User enters new search.
			\begin{enumerate}[{VP1.1}]
				
				% Viewpoint 1
				\item Users
				\begin{enumerate}
					\item The system shall be able to display the resulting locations on a map.
					\item The system shall be able to ask the user questions.
					\item The system shall allow the user to answer yes or no to questions.
					\item The system shall determine the closest locations, environment and place.
					\item The system shall allow each expert� to determine the answer given the set of questions and their answers.
				\end{enumerate}
				
				%Viewpoint 2
				\item Establishments
				\begin{enumerate}
					\item Not Applicable
				\end{enumerate}
				
				%Viewpoint 3
				\item TA/Prof (Management role)
				\begin{enumerate}
					\item Not Applicable
				\end{enumerate}
				
				%Viewpoint 4
				\item IT Maintenance
				\begin{enumerate}
					\item Not Applicable
				\end{enumerate}
				
				%Viewpoint 5
				\item Front End Developers
				\begin{enumerate}
					\item Not Applicable
				\end{enumerate}
				
				%Viewpoint 6
				\item Back End Developers
				\begin{enumerate}
					\item Not Applicable
				\end{enumerate}
			\end{enumerate}
			
			
			%Business Event 2
			\item An unlisted establishment requests to be included in the application.
			\begin{enumerate}[{VP2.1}]
				
				%Viewpoint 1
				\item Users
				\begin{enumerate}
					\item Not Applicable 
				\end{enumerate}
				
				%Viewpoint 3
				\item Establishments
				\begin{enumerate}
					\item The system shall be able to display the resulting locations on a map.
				\end{enumerate}
				
				%Viewpoint 4
				\item TA/Prof (Management role)
				\begin{enumerate}
					\item The system shall allow developers to easily add or remove questions or locations.
				\end{enumerate}
				
				%Viewpoint 5
				\item IT Maintenance
				\begin{enumerate}
					\item The system shall allow developers to easily add or remove questions or locations.
				\end{enumerate}
				
				%Viewpoint 6
				\item Front End Developers
				\begin{enumerate}
					\item The system shall be able to display the resulting locations on a map.				\end{enumerate}
				
				%Viewpoint 7
				\item Back End Developers
				\begin{enumerate}
					\item The system shall allow developers to easily add or remove questions and locations.
				\end{enumerate}
			\end{enumerate}
			
			%Business Event 3
			\item Updates or maintenance of the app is required.
			\begin{enumerate}[{VP3.1}]
				
				%Viewpoint 1
				\item Users
				\begin{enumerate}
					\item The system shall inform the user if an update or maintenance is required.
				\end{enumerate}
				
				%Viewpoint 2
				\item Establishments
				\begin{enumerate}
					\item Not Applicable
				\end{enumerate}
				
				%Viewpoint 3
				\item TA/Prof (Management role)
				\begin{enumerate}
					\item The system shall allow the user to flag or report any inappropriate answers or questions.
				\end{enumerate}
				
				%Viewpoint 4
				\item IT Maintenance
				\begin{enumerate}
					\item  The system shall developers to easily add or remove questions and locations.
				\end{enumerate}
				
				%Viewpoint 5
				\item Front End Developers
				\begin{enumerate}
					\item The system shall developers to easily add or remove questions and locations.
					\item The system shall have 3 experts that will be able to be easily swapped out.
				\end{enumerate}
				
				%Viewpoint 6
				\item Back End Developers
				\begin{enumerate}
					\item The system shall developers to easily add or remove questions and locations.
				\end{enumerate}
			\end{enumerate}
			
			%Business Event 4
			\item Adding and Removing Functionality
			\begin{enumerate}[{VP4.1}]
				
				%Viewpoint 1
				\item Users
				\begin{enumerate}
					\item Not Applicable
				\end{enumerate}

				%Viewpoint 2
				\item Establishments
				\begin{enumerate}
					\item Not Applicable
				\end{enumerate}
				
				%Viewpoint 3
				\item TA/Prof (Management role)
				\begin{enumerate}
					\item Not Applicable
				\end{enumerate}
				
				%Viewpoint 4
				\item IT Maintenance
				\begin{enumerate}
					\item The system shall developers to easily add or remove questions and locations.
					\item The system shall not effect the integrity of the existing information.
				\end{enumerate}
				
				%Viewpoint 5
				\item Front End Developers
				\begin{enumerate}
					\item The system shall developers to easily add or remove questions and locations.
					\item The system shall not effect the integrity of the existing information.
				\end{enumerate}
				
				%Viewpoint 6
				\item Back End Developers
				\begin{enumerate}
					\item The system shall developers to easily add or remove questions and locations.
					\item The system shall not effect the integrity of the existing information.
				\end{enumerate}
			\end{enumerate}
			
			%Business Event 5
			\item User flags an incorrect or inappropriate search or result.
			\begin{enumerate}[{VP5.1}]
				
				%Viewpoint 1
				\item Users
				\begin{enumerate}
					\item The system shall allow the user to flag or report any inappropriate answers or questions.
				\end{enumerate}
				
				%Viewpoint 2
				\item Establishments
				\begin{enumerate}
					\item Not Applicable
				\end{enumerate}
				
				%Viewpoint 3
				\item TA/Prof (Management role)
				\begin{enumerate}
					\item The system shall allow the user to flag or report any inappropriate answers or questions.
				\end{enumerate}
				
				%Viewpoint 4
				\item IT Maintenance
				\begin{enumerate}
					\item The system shall developers to easily add or remove questions and locations.
				\end{enumerate}
				
				%Viewpoint 5
				\item Front End Developers
				\begin{enumerate}
					\item The system shall developers to easily add or remove questions and locations.
				\end{enumerate}
				
				%Viewpoint 6
				\item Back End Developers
				\begin{enumerate}
					\item The system shall developers to easily add or remove questions and locations.
				\end{enumerate}
			\end{enumerate}
			
		\end{enumerate}
		
		\section{Non-Functional Requirements}
		\label{sec:non-functional_requirements}
		% Begin Section
		
		
		\subsection{Look and Feel Requirements}
		\label{sub:look_and_feel_requirements}
		\begin{enumerate}
			\item
			The system shall be usable on the first use by any user over the age of ten.
			\item
			The system shall display any information in a visual way. % Expand
			\item
			The system shall show a graphical view of locations and \textbf{AOI}s.
			\item
			The system shall include a tutorial to teach new users correct usage of the application.
		\end{enumerate}
		
		\subsubsection{Appearance Requirements}
		\label{ssub:appearance_requirements}
		% Begin SubSubSection
		\begin{enumerate}[{LF}1. ]
			\item 
			The game will appear in a pleasant manner suitable for all demographics.
			\item
			The system shall use modern graphical user interface libraries and techniques.
		\end{enumerate}
		% End SubSubSection
		
		\subsubsection{Style Requirements}
		\label{ssub:style_requirements}
		% Begin SubSubSection
		\begin{enumerate}[{LF}1. ]
			\item 
			The style of the game's GUI will not distract from the essence of the game.
			\item
			The game will be playable by people affected by colour vision deficiencies.
			\item
			The game will be playable by people affected by hearing deficiencies.
		\end{enumerate}
		% End SubSubSection
		
		% End SubSection
		
		\subsection{Usability and Humanity Requirements}
		\label{sub:usability_and_humanity_requirements}
		
		
		\subsubsection{Ease of Use Requirements}
		\label{ssub:ease_of_use_requirements}
		% Begin SubSubSection
		\begin{enumerate}[{UH}1.]
			\item 
			The system shall provide questions that are easy to understand.
			\item
			The system shall be easy to navigate and understand the layout.
		\end{enumerate}
		% End SubSubSection
		
		\subsubsection{Personalization and Internationalization Requirements}
		\label{ssub:personalization_and_internationalization_requirements}
		% Begin SubSubSection
		\begin{enumerate}[{UH}1. ]
			\item
			The system shall be able to set fonts and colours to the users preference.
		\end{enumerate}
		% End SubSubSection
		
		\subsubsection{Learning Requirements}
		\label{ssub:learning_requirements}
		% Begin SubSubSection
		\begin{enumerate}[{UH}1. ]
			\item
			The system will provide a set of instructions describing the game's rules and objectives.
		\end{enumerate}
		% End SubSubSection
		
		\subsubsection{Understandability and Politeness Requirements}
		\label{ssub:understandability_and_politeness_requirements}
		% Begin SubSubSection
		\begin{enumerate}[{UH}1. ]
			\item 
			The questions asked in the game will be simple and focus on one specific point. %% Research lawsuits about language used in referendums for more details to add to this.
			\item
			The questions asked in the game will not include offensive, insensitive or immature remarks.
		\end{enumerate}
		% End SubSubSection
		
		\subsubsection{Accessibility Requirements}
		\label{ssub:accessibility_requirements}
		% Begin SubSubSection
		\begin{enumerate}[{UH}1. ]
			\item 
			The game will be playable for people affected by colour vision deficiencies.
			\item
			The game will be playable for people affected by hearing deficiencies.
			\item
			The system will be able to read questions aloud to the user.
		\end{enumerate}
		% End SubSubSection
		
		% End SubSection
		
		\subsection{Performance Requirements}
		\label{sub:performance_requirements}
		
		
		\subsubsection{Speed and Latency Requirements}
		\label{ssub:speed_and_latency_requirements}
		% Begin SubSubSection
		\begin{enumerate}
			\item
			Any operation that does not require use of the internet will respond within 2 seconds.
			\item
			Any operation that does require use of the internet will respond within 30 seconds.
		\end{enumerate}
		% End SubSubSection
		
		\subsubsection{Safety-Critical Requirements}
		\label{ssub:safety_critical_requirements}
		% Begin SubSubSection
		There are no safety critical requirements for this application.
		% End SubSubSection
		
		\subsubsection{Precision or Accuracy Requirements}
		\label{ssub:precision_or_accuracy_requirements}
		% Begin SubSubSection
		\begin{enumerate}
			\item
			The system shall predict the correct location with 75\% accuracy.
			\item
			The system shall predict the user's environment 90\% of the time.
			\item
			The system shall predict landmarks 80\% of the time.
		\end{enumerate}
		% End SubSubSection
		
		\subsubsection{Reliability and Availability Requirements}
		\label{ssub:reliability_and_availability_requirements}
		% Begin SubSubSection
		\begin{enumerate}[{PR}1. ]
			\item
			The game requires internet access to play.
		\end{enumerate}
		% End SubSubSection
		
		\subsubsection{Robustness or Fault-Tolerance Requirements}
		\label{ssub:robustness_or_fault_tolerance_requirements}
		% Begin SubSubSection
		\begin{enumerate}[{PR}1. ]
			\item 
			The system shall include a wide variety of locations and monuments.
		\end{enumerate}
		% End SubSubSection
		
		\subsubsection{Capacity Requirements}
		\label{ssub:capacity_requirements}
		% Begin SubSubSection
		\begin{enumerate}[{PR}1. ]
			\item 
			The system should be able to respond to at least 20 client queries at a time.
		\end{enumerate}
		% End SubSubSection
		
		\subsubsection{Scalability or Extensibility Requirements}
		\label{ssub:scalability_or_extensibility_requirements}
		% Begin SubSubSection
		\begin{enumerate}
			\item
			The system should be able to add new \textbf{AOI}s easily.
		\end{enumerate}
		% End SubSubSection
		
		\subsubsection{Longevity Requirements}
		\label{ssub:longevity_requirements}
		% Begin SubSubSection
		\begin{enumerate}[{PR}1. ]
			\item There are no longevity requirements for this application.
		\end{enumerate}
		% End SubSubSection
		
		% End SubSection
		
		\subsection{Operational and Environmental Requirements}
		\label{sub:operational_and_environmental_requirements}
		\begin{enumerate}
			\item
			The system shall check to ensure that there is a sufficient internet connection.
			\item
			The system shall ensure that any servers are up for 80\% of the time.
		\end{enumerate}
		
		\subsubsection{Expected Physical Environment}
		\label{ssub:expected_physical_environment}
		% Begin SubSubSection
		The application's expected physical environment will be anywhere that a mobile phone can operate. Typically users will not interact with the system in an emergency or other circumstantial event.
		% End SubSubSection
		
		\subsubsection{Requirements for Interfacing with Adjacent Systems}
		\label{ssub:requirements_for_interfacing_with_adjacent_systems}
		% Begin SubSubSection
		\begin{enumerate}[{OE}1. ]
			\item 
			The system should provide an interface to connect with adjacent systems.
		\end{enumerate}
		% End SubSubSection
		
		\subsubsection{Productization Requirements}
		\label{ssub:productization_requirements}
		% Begin SubSubSection
		\begin{enumerate}[{OE}1. ]
			\item There are no productization requirements for this application.
		\end{enumerate}
		% End SubSubSection
		
		\subsubsection{Release Requirements}
		\label{ssub:release_requirements}
		% Begin SubSubSection
		\begin{enumerate}[{OE}1. ]
			\item The system shall be fully functional and have been tested for security flaws before release.
			\item The system shall be available to download in the \textit{Google Play Store}.
			\item The system will be available for direct download.
		\end{enumerate}
		% End SubSubSection
		
		% End SubSection
		
		\subsection{Maintainability and Support Requirements}
		\label{sub:maintainability_and_support_requirements}
		% Begin SubSection
		\subsubsection{Maintenance Requirements}
		\label{ssub:maintenance_requirements}
		% Begin SubSubSection
		\begin{enumerate}
			\item 
			The system shall be easy to update.
			\item
			The system shall be able to easily swap out modules.
		\end{enumerate}
		% End SubSubSection
		
		\subsubsection{Supportability Requirements}
		\label{ssub:supportability_requirements}
		% Begin SubSubSection
		\begin{enumerate}
			\item 
			The system shall be able to run on at least 90\% of Android devices.
		\end{enumerate}
		% End SubSubSection
		
		\subsubsection{Adaptability Requirements}
		\label{ssub:adaptability_requirements}
		% Begin SubSubSection
		\begin{enumerate}
			\item
			The system can easily be implemented on new operating systems with little change.
		\end{enumerate}
		% End SubSubSection
		
		% End SubSection
		
		\subsection{Security Requirements}
		\label{sub:security_requirements}
		\begin{enumerate}
			\item
			Any information that enters or exits the system shall be encrypted.
			\item
			The system will not store or transmit information related to the user's location.
			\item
			The system will not store usernames or passwords.
		\end{enumerate}
		
		\subsubsection{Access Requirements}
		\label{ssub:access_requirements}
		% Begin SubSubSection
		\begin{enumerate}[{SR}1. ]
			\item
			The system should be able to operate anywhere that a data or wifi connection is available.
		\end{enumerate}
		% End SubSubSection
		
		\subsubsection{Integrity Requirements}
		\label{ssub:integrity_requirements}
		% Begin SubSubSection
		\begin{enumerate}[{SR}1. ]
			\item 
			The system shall not allow any data to be modified by any algorithms.
		\end{enumerate}
		% End SubSubSection
		
		\subsubsection{Privacy Requirements}
		\label{ssub:privacy_requirements}
		% Begin SubSubSection
		\begin{enumerate}[{SR}1. ]
			\item 
			The system shall not store any user's location data.
			\item
			The system shall not store any user passwords.
			\item
			The system shall not release information to outside parties about a specific business.
		\end{enumerate}
		% End SubSubSection
		
		\subsubsection{Audit Requirements}
		\label{ssub:audit_requirements}
		% Begin SubSubSection
		\begin{enumerate}[{SR}1. ]
			\item 
			Following a security audit at least 80\% of the changes necessary shall be implemented.
		\end{enumerate}
		% End SubSubSection
		
		\subsubsection{Immunity Requirements}
		\label{ssub:immunity_requirements}
		% Begin SubSubSection
		\begin{enumerate}[{SR}1. ]
			\item 
			The system shall not lend itself vulnerable to attacks or intruders.
			\item
			The system shall only allow those with proper clearances to modify the data.
		\end{enumerate}
		% End SubSubSection
		
		% End SubSection
		
		\subsection{Cultural and Political Requirements}
		\label{sub:cultural_and_political_requirements}
		% Begin SubSection
		
		% business event: user tries to update database from old versioned app
		\subsubsection{Cultural Requirements}
		\label{ssub:cultural_requirements}
		% Begin SubSubSection
		\begin{enumerate}
			\item
			The system shall ensure that culturally significant \textbf{AOI}s are presented in a respectful manner.
		\end{enumerate}
		% End SubSubSection
		
		\subsubsection{Political Requirements}
		\label{ssub:political_requirements}
		% Begin SubSubSection
		\begin{enumerate}
			\item
			The system shall show no bias towards any political party or related organization.
			\item
			The system will not endorse or associate with any political group or government.
		\end{enumerate}
		% End SubSubSection
		
		% End SubSection
		\subsection{Legal Requirements}
		\label{sub:legal_requirements}
		
		\subsubsection{Compliance Requirements}
		\label{ssub:compliance_requirements}
		% Begin SubSubSection
		\begin{enumerate}[{LR}1. ]
			\item
			The system shall operate within the laws set by Canada and the United States of America.
			\item
			The system shall comply with any libraries offered by Google Maps.
		\end{enumerate}
		% End SubSubSection
		
		\subsubsection{Standards Requirements}
		\label{ssub:standards_requirements}
		% Begin SubSubSection
		\begin{enumerate}[{LR}1. ]
			\item
			The system shall meet the standards set by Management (Professor and TAs).
			\item
			The system shall meet the standards set by ISO/IEC 12207.
		\end{enumerate}
		% End SubSubSection
		
		% End SubSection
		
		% End Section
		
		\clearpage{}
		\appendix
		\section{Division of Labour}
		\label{sec:division_of_labour}
		% Begin Section
		The following includes the division of labour for Deliverable 1:

		\begin{longtable}{| p{.30\textwidth} | p{.70\textwidth} |}
		\hline
		\textbf {Team Member} & \textbf{Contributions}\\ 
		\hline
		Gabriel Lopez de Leon & Please enter your stuff\\
		\hline
		Maxwell Moore & Please enter your stuff\\
		\hline
		Curtis Milo & Generated and refined business events, view points, functional and non- functional requirements as well as constraints. Created list of assumptions, dependancies and apportioning of requirements. \\ 
		\hline
		Alexandra Rahman & Generated and refined business events, view points, functional and non-functional requirements as well as edited the final document. Assigned each business event and viewpoint a functional requirement if applicable.\\
		\hline
		Connor Sheehan & Please enter your stuff\\
		\hline
		
		\caption{Division of Labour}
		\end{longtable}
		
		
		
		
		% End Section
		
		
		%% Important notes from TAs, commented out
		\iffalse
		\newpage
		\section*{IMPORTANT NOTES}
		\begin{itemize}
			\item Be sure to include all sections of the template in your document regardless whether you have something to write for each or not
			\begin{itemize}
				\item If you do not have anything to write in a section, indicate this by the \emph{N/A}, \emph{void}, \emph{none}, etc.
			\end{itemize}
			\item Uniquely number each of your requirements for easy identification and cross-referencing
			\item Highlight terms that are defined in Section~1.3 (\textbf{Definitions, Acronyms, and Abbreviations}) with \textbf{bold}, \emph{italic} or \underline{underline}
			\item For Deliverable 1, please highlight, in some fashion, all (you may have more than one) creative and innovative features. Your creative and innovative features will generally be described in Section~2.2 (\textbf{Product Functions}), but it will depend on the type of creative or innovative features you are including.
		\end{itemize}
		\fi
		
	\end{document}
	%------------------------------------------------------------------------------