\documentclass[titlepage]{article}

% Imported Packages
%------------------------------------------------------------------------------
\usepackage{amssymb}
\usepackage{amstext}
\usepackage{amsthm}
\usepackage{amsmath}
\usepackage{enumerate}
\usepackage{fancyhdr}
\usepackage[margin=1in]{geometry}
\usepackage{graphicx}
\usepackage{extarrows}
\usepackage{setspace}
\usepackage{longtable}
%------------------------------------------------------------------------------

% Header and Footer
%------------------------------------------------------------------------------
\pagestyle{fancy}  
\lhead{Group 2}
\chead{SFWR 3A04}
\rhead{Detailed Architectural Design} 
\renewcommand\headrulewidth{0.4pt}                                      
\renewcommand\footrulewidth{0.4pt}                                    
%------------------------------------------------------------------------------

% Title Details
%------------------------------------------------------------------------------
\title{\Huge{\textbf{Detailed Architectural Design\\21 Questions}}\vspace{9cm}}
\author{\textit{Gabriel Lopez de Leon} - lopezdg - 1310514\\\textit{Curtis Milo} - milocj - 1305877\\\textit{Maxwell Moore} - moorem8 - 1320009\\\textit{Alexandra Rahman} - rahmaa25 - 1305735\\\textit{Connor Sheehan} - sheehacg - 1330964}
\date{Monday, March 28\textsuperscript{th}, 2016}                               
%------------------------------------------------------------------------------

% Document
%------------------------------------------------------------------------------
\begin{document}

\maketitle	

\section{Introduction}
\label{sec:introduction}
% Begin Section

This document is called the detailed architectural design document and it will be giving the stakeholders an in depth look at how the project, \textit{21 Questions}, is designed. More precisely, this document outlines the details concerning each class as well as how classes interact with one another.

\subsection{Purpose}
\label{sub:purpose}

% Begin SubSection

The detailed architectural design document outlines the functionality of the system through the State Chart diagrams, and focuses on the detail of each class through the Detailed Class diagrams. The interactions among the classes, more specifically the messages passed between classes, will be defined in the Sequence diagrams.  The main purpose of this document is to explain in detail the software system to be developed, which in this case is the application, \textit{21 Questions}. Through the use of the various diagrams mentioned before, all components of the product and their relationships with each other are revealed in more detail. The main target audience for this document is the software developers as they need to see, in greater detail, how the system and its modules interact and how they are designed to work in relation to each other.

% End SubSection

\subsection{System Description}
\label{sub:system_description}

% Begin SubSection

\textit{21 Questions} is an android application that can be used as a location identifier whose intended use is for any user above the age of ten. The application requires minimal training, experience or technical expertise to use, and can be easily picked up and used by anyone. 21 Questions is a simple game that asks the user a series of twenty-one polar or binary questions to try to identify their area of interest. In this game the area of interest is limited to an establishment, building, place, or effigy with a focus on locations only with an end goal of displaying the result through Google Maps.

% End SubSection

\subsection{Overview}
\label{sub:overview}

% Begin SubSection

This document will not only outline the design, but the details of the \textit{21 Questions} application from an implementation perspective. The document will begin from state perspective, in more detail how one gets from one state to another as well as which states follow each other. This is portrayed in the State Charts. This is followed by a sequence overview depicting how each class passes messages to one another. Finally the details of each class are 
illustrated in the Detailed Class diagram. These sections highlight the classes in greater detail, including how each class interacts with one another. This document easily lends itself to being a model for the implementation.

% End SubSection

% End Section

\section{State Charts for Controller Classes}
\label{sec:state_charts_for_controller_classes}
% Begin Section
The following section contains state charts for the controller classes in the \textit{21 Questions} application. Figures \ref{sc:gui} and \ref{sc:question} are the state charts for the GUI and Question Controller classes, respectively.

\begin{figure}
\includegraphics[scale=0.8]{sd_guicontroller}
\caption{State chart for the GUI Controller class.}\label{sc:gui}
\end{figure}

\begin{figure}
\includegraphics[scale=0.8]{sd_questioncontroller}
\caption{State chart for the Question Controller class.}\label{sc:question}
\end{figure}
% End Section

\section{Sequence Diagrams}
\label{sec:sequence_diagrams}
% Begin Section
%This section should provide a sequence diagram for each use case of your application. *** NOTE: MAKE MORE OF THESE -cs ***
Figure \ref{sd:main} contains a sequence diagram for the primary use case of the application.

\begin{figure}
\includegraphics[scale=0.5]{Sequence_Diagram}
\caption{Sequence diagram for the primary use case.}\label{sd:main}
\end{figure}
% End Section

\section{Detailed Class Diagram}
\label{sec:detailed_class_diagram}
% Begin Section
This section should provide a detailed class diagram for your application.
% End Section

\pagebreak
\appendix
\section{Division of Labour}
\label{sec:division_of_labour}
% Begin Section

\begin{longtable}{| p{.25\textwidth} | p{.70\textwidth} |}
			\hline
			\textbf {Team Member} & \textbf{Contributions}\\ 
			\hline
			Gabriel Lopez de Leon & Insert your contribution here.
			\\
			\hline
			Maxwell Moore & Insert your contribution here.
			\\
			\hline
			Curtis Milo & Insert your contribution here.
			\\ 
			\hline
			Alexandra Rahman & Insert your contribution here.
			\\
			\hline
			Connor Sheehan & Collaborated on design of architecture. Generated state charts.
			\\
			\hline
			
			\caption{Division of Labour}
	\end{longtable}



\noindent\begin{tabular}{ll}\\
	\makebox[2.5in]{\hrulefill} & \makebox[2.5in]			{\hrulefill}\\
	Gabriel Lopez de Leon & Date\\[8ex]% adds space between the two sets of signatures
	\makebox[2.5in]{\hrulefill} & \makebox[2.5in]			{\hrulefill}\\
	Curtis Milo & Date\\[8ex]
	\makebox[2.5in]{\hrulefill} & \makebox[2.5in]			{\hrulefill}\\
	Maxwell Moore & Date\\[8ex]
	\makebox[2.5in]{\hrulefill} & \makebox[2.5in]			{\hrulefill}\\
	Alexandra Rahman & Date\\[8ex]
	\makebox[2.5in]{\hrulefill} & \makebox[2.5in]			{\hrulefill}\\
	Connor Sheehan & Date\\
		\end{tabular}



% End Section
\iffalse
\newpage
\section*{IMPORTANT NOTES}
\begin{itemize}
	\item You do \underline{NOT} need to provide a text explanation of each diagram; the diagram should speak for itself
	\item Please document any non-standard notations that you may have used
	\begin{itemize}
		\item \emph{Rule of Thumb}: if you feel there is any doubt surrounding the meaning of your notations, document them
	\end{itemize}
	\item Some diagrams may be difficult to fit into one page
	\begin{itemize}
		\item It is OK if the text is small but please ensure that it is readable when printed
		\item If you need to break a diagram onto multiple pages, please adopt a system of doing so and throughly explain how it can be reconnected from one page to the next; if you are unsure about this, please ask me
	\end{itemize}
	\item Please submit the latest version of Deliverable 1 and Deliverable 2 with Deliverable 3
	\begin{itemize}
		\item They do not have to be a freshly printed versions; the latest marked versions are OK
	\end{itemize}
	\item If you do \underline{NOT} have a Division of Labour sheet, your deliverable will \underline{NOT} be marked
\end{itemize}
\fi

\end{document}
%------------------------------------------------------------------------------