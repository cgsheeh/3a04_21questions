\documentclass[titlepage]{article}

% Imported Packages
%------------------------------------------------------------------------------
\usepackage{amssymb}
\usepackage{amstext}
\usepackage{amsthm}
\usepackage{amsmath}
\usepackage{enumerate}
\usepackage{fancyhdr}
\usepackage[margin=1in]{geometry}
\usepackage{graphicx}
%\usepackage{extrarrows}
\usepackage{setspace}
\usepackage{longtable}
\usepackage{float}
\floatplacement{figure}{H}
%------------------------------------------------------------------------------

% Header and Footer
%------------------------------------------------------------------------------
\pagestyle{fancy}
\lhead{Group 2}
\chead{SFWR 3A04}
\rhead{High-Level Architectural Design} 
\renewcommand\headrulewidth{0.4pt}                                      
\renewcommand\footrulewidth{0.4pt}                                    
%------------------------------------------------------------------------------

% Title Details
%------------------------------------------------------------------------------
\title{\Huge{\textbf{High-Level Architectural Design\\21 Questions}}\vspace{9cm}}
\author{\textit{Gabriel Lopez de Leon} - lopezdg - 1310514\\\textit{Curtis Milo} - milocj - 1305877\\\textit{Maxwell Moore} - moorem8 - 1320009\\\textit{Alexandra Rahman} - rahmaa25 - 1305735\\\textit{Connor Sheehan} - sheehacg - 1330964}
\date{March 7\textsuperscript{th}, 2016}                               
%------------------------------------------------------------------------------

% Document
%------------------------------------------------------------------------------
\begin{document}
\maketitle	

\tableofcontents
\listoffigures
\listoftables
\newpage

\section{Introduction}
\label{sec:introduction}
% Begin Section

This document is called the High-Level Architectural Design document and it will be giving the stakeholders a broad overview of how the project 21 Questions is designed and will be organized. 

\subsection{Purpose}
\label{sub:purpose}
% Begin SubSection
The High-Level Architectural Design document outlines the functionality of the system through Use Case diagrams, and details the key classes of the system and how they relate via the use of an Analysis Class Diagram. The main purpose of this document is to explain in detail the software system to be developed, which in this case is the application, 21 Questions. Through the use of the various diagrams mentioned before, the main components of the product and their relationships with each other are shown. The main target audience for this document is the software developers as they need to see, in greater detail, how the system and its modules interact and how they are designed to work in relation to each other.
% End SubSection

\subsection{System Description}
\label{sub:system_description}
% Begin SubSection
21 Questions is an android application that can be used as a location identifier whose intended use is for any user above the age of ten. The application requires minimal training, experience or technical expertise to use, and can be easily picked up and used by anyone. 21 Questions is a simple games that asks the user a series of twenty-one polar or binary questions to try to identify their area of interest. In this game the area of interest is limited to an establishment, building, place, or effigy with a focus on locations only with an end goal of displaying the result through Google Maps.
% End SubSection

\subsection{Overview}
\label{sub:overview}
% Begin SubSection
This document will outline the design of the \textit{21 Questions} application from an architectural perspective. The document will begin from a use case outlook, outlining application functionality from a practical point of view and taking different actors and stakeholders into consideration. Next, an analysis class diagram and associated interpretation details is outlined, to specify application behaviours and resources in a modularized form. Following this section is a detailed architectural design as well as a set of class responsibility collaboration cards. These sections specify modules in greater detail, including interfaces to be implemented in the future. The order of these sections reflects a systematic progression from requirements to a more easily constructed application.
% End SubSection
% End Section



\newpage
\section{Use Case Diagram}
\label{sec:use_case_diagram}
% Begin Section
\begin{enumerate}[a)]
	%Business Event 1
	\item User wants to enter new search (Figure \ref{usecase:search}).
		%Scenario
		\begin{enumerate}[{BE1}.1]
			\item The user presses the start button on the start screen.
			\item The system will respond by bringing them to the question screen and asking them a question.
			\item The user will answer yes, no or undecided to the question.
			\item The system will ask the user another question.
			\item After 21 questions, the system will respond by displaying a map to the user.
			\item The user will hit done.
			\item The system will ask if this was the location the user had in mind.
			\item The user will answer yes it was or no it was not.
		\end{enumerate}
	%UC1
	\begin{center}
		\begin{figure}[H]
			\includegraphics[scale=0.6]{BE1}
			\caption{Use Case for BE1}\label{usecase:search}
		\end{figure}
	\end{center}

	%Business Event 2
	\item An unlisted establishment requests to be included in the application (Figure \ref{usecase:new_esta}).
		%Scenario
		\begin{enumerate}[{BE2}.1]
			\item A new business opens or a business opens a new location.
			\item The business or establishment contacts the company to inform them they wish to be added to the system.
			\item The IT specialists will send them a form for the business to fill out.
			\item The business will return the form to the company.
			\item The IT will verify that the information is valid and add it to the system.
			\item The business is added.
		\end{enumerate}
	%UC2	
	\begin{center}
		\begin{figure}[H]
			\includegraphics[scale=0.6]{BE2}
			\caption{Use Case for BE2}\label{usecase:new_esta}
		\end{figure}
	\end{center}


	%Business Event 3
	\item Updates or maintenance of the application is required (Figure \ref{usecase:updates}).
		%Scenario
		\begin{enumerate}[{BE3}.1]
			\item Internal management states an issue and requests an update/ maintenance.
			\item Update/ maintenance is given a priority.
			\item The IT specialist notifies the users that the system will update and be shut down for a certain period of time, if necessary.
			\item The system will disconnect.
			\item The necessary changes shall be made by the IT specialists.
			\item The user will be notified if they need to update the application version.
		\end{enumerate}
	%UC3
	\begin{center}
		\begin{figure}[H]
			\includegraphics[scale=0.6]{BE3}
			\caption{Use Case for BE3}\label{usecase:updates}
		\end{figure}
	\end{center}
	
	%Business Event 4
	\item Management requests implementation or change of experts (Figure \ref{usecase:change_expert}).
		%Scenario
		\begin{enumerate}[{BE4}.1]
			\item Internal management states a function needs to be changed(added or removed).
			\item The change is given a priority.
			\item Function will be added or removed.
			\item A small focus group is selected.
			\item A survey is created.
			\item Update is released to focus group.
			\item Survey is sent to focus group.
			\item The update will be released depending on the results of the survey.
			\item If the survey results are not favourable, the function will be under review and released again to the focus group (repeat steps 5-9). Otherwise the update is released to the general public and the user is notified that they need to update app version.
		\end{enumerate}
	%UC4
	\begin{center}
		\begin{figure}[h!]
			\includegraphics[scale=0.6]{BE4}
			\caption{Use Case for BE4}\label{usecase:change_expert}
		\end{figure}
	\end{center}

\newpage
	%Business Event 5
	\item User flags an incorrect or inappropriate search or result (Figure \ref{usecase:badsearch}).
		%Scenario
		\begin{enumerate}[{BE5}.1]
			\item A business, user or internal management recognizes that the content is inappropriate.
			\item The content shall automatically be hidden from other users.
			\item The content is given a priority.
			\item The content is put into a priority queue.
			\item An IT will review the content and make appropriate changes (remove if necessary).
		\end{enumerate}
	%UC5
	\begin{center}
		\begin{figure}[H]
			\includegraphics[scale=0.6]{BE5}
			\caption{Use Case for BE5}\label{usecase:badsearch}
		\end{figure}
	\end{center}
	
	\end{enumerate}
% End Section

\newpage

\section{Analysis Class Diagram}
\label{sec:analysis_class_diagram}
% Begin Section
This section should provide an analysis class diagram for your application.

\label{sub:system_architecture}
% Begin SubSection
	\begin{center}
		\begin{figure}[H]
			\includegraphics[scale=0.5]{analysis_class_diagram}
			\caption{Analysis Class Diagram}\label{diagram:analysisclass}
		\end{figure}
	\end{center}
	
% End Section


\section{Architectural Design}
\label{sec:architectural_design}
% Begin Section
The desired system architecture for solving this issue is the blackboard architecture because of experts are able to identify different aspects of the area of interest, these module can be swapped out if needed to give a more accurate result. Each expert will be there own ADT on the front end of the application so that it will be easy to provide an interface for experts so that we increase the modularity of the design. Each expert will be able to communicate to there own database, the database is essential to ensure that multiple clients will have access to the same information without having to preform updates.

\subsection{System Architecture}

Figure \ref{diagram:analysisclass} contains the analysis class diagram. The diagram is designed so that way each entity or boundary class will communication there controller and the controllers will communication between each other.The design is set up into three separate units of: interface, logic and data. The system design is a black board model with a data store, with the experts all using the same interface so that they can properly communicate to the question controller class. The system will have two controllers, one for handling the pages of the system and the other for controlling which expert can ask questions. Each expert will create a guess of what they might think is the item in which the user is thinking of. The controller will ask one expert at a time to provide a question to ask the user, based on the users response will either prove that is not the thing we are looking for or that it still might be. The experts will then provide there best guess to the controller, the controller will then compile these answers into a search in which will be displayed on a map for the user. In this way our system will not rely on any one expert, making it easy to add and remove experts in order to provide a better result.

\subsection{Subsystems}
\label{sub:subsystems}
% Begin SubSection
The database controller will deal with different clients asking for unique information. This entitles that it will process any information, run the query that is needed, and send the information that is related to the problem. \\ \\
The next tier to the system is the logical layer. This layer is responsible for the control flow of the game, ensuring that each expert entity will properly get to ask there respective questions. The controller will deal with receiving the information that is passed in from the GUI controller and dealing with incoming information from the Data layer. Using the guesses form each of the experts to create a final solution to the problem.\\ \\
Finally the GUI controller, will with ensuring the proper screen is shown. The controller receive, process and send information from the screens to the question controller. The receive information from the logic controller to use in the questions or map boundary class. The start screen, which will show the initial menu a user will be viewing, this module will take user input to start a game or change the setting of the game. The question controller will display the questions for the user to answer and will pass off this information the the GUI controller. The setting screen will provide a sub-menu which would allow a user to change the functionality of the system in regards to: accessibility, font colors, sizes, and preferences. finally the map screen will be where we provide the user with a map of what the final result is and will take feedback from the user if it is correct or incorrect.
% End SubSection

% End Section
	
\newpage	
\section{Class Responsibility Collaboration (CRC) Cards}
\label{sec:class_responsibility_collaboration_crc_cards}
% Begin Section
This section should contain all of your CRC cards.
	
	%CRC for Expert Controller 
	\begin{longtable}{| p{.70\textwidth} | p{.30\textwidth} |}
			\hline
			 \multicolumn{2}{|l|}{\textbf{Class Name: Expert Controller}} \\
			\hline
			\textbf{Responsibility:} & \textbf{Collaborators:} \\
			\hline
				\begin{itemize}
					\item Deals with messages passed from the Question Controller
					\item Passes desired questions to the Question Controller
					\item Creates and deletes tables that relate to experts
					\item Accesses the information by performing a query on the tables
					\item Adds information to tables
					\item Modifies information in the tables
				\end{itemize} & 
				\begin{itemize}
					\item Establishment Table
					\item Environment Table
					\item Location Table
				\end{itemize} 
				\\
			\hline
		\caption{CRC for Expert Controller}
	\end{longtable}
	
	
	%CRC for Question Controller 
	\begin{longtable}{| p{.70\textwidth} | p{.30\textwidth} |}
			\hline
			 \multicolumn{2}{|l|}{\textbf{Class Name: Question Controller}} \\
			\hline
			\textbf{Responsibility:} & \textbf{Collaborators:} \\
			\hline
				\begin{itemize}
					\item Requests questions from the Expert Controller based on the expert's needs
					\item Accesses questions that have already been asked as well as the answers to said questions
					\item Keeps track of the overall state of the game. This includes the number of questions total that have been asked
					\item Provides a solution based on the experts' best guesses
					\item Process answers from the GUI Controller
					\item Provides the next question for the GUI Controller to display
				\end{itemize} & 
				\begin{itemize}
					\item Expert Controller
					\item Facts Data
					\item GUI Controller
				\end{itemize} 
				\\
			\hline
		\caption{CRC for Question Controller}
	\end{longtable}
	
\newpage	
	%CRC for GUI Controller 
	\begin{longtable}{| p{.70\textwidth} | p{.30\textwidth} |}
			\hline
			 \multicolumn{2}{|l|}{\textbf{Class Name: Graphics User Interface Controller}} \\
			\hline
			\textbf{Responsibility:} & \textbf{Collaborators:} \\
			\hline
				\begin{itemize}
					\item Controls which boundary class the user can interact with
					\item Processes and verifies information to send to the Question Controller
					\item Receives information from the Question Controller to display information or direct to answer
					\item Tracks the current and past state
				\end{itemize} & 
				\begin{itemize}
					\item Question Controller
					\item Map Screen
					\item Start Screen
					\item Question Screen
					\item Setting Screen
				\end{itemize} 
				\\
			\hline
		\caption{CRC for Graphics User Interface Controller}
	\end{longtable}
	
	%CRC for Start Screen
	\begin{longtable}{| p{.70\textwidth} | p{.30\textwidth} |}
			\hline
			 \multicolumn{2}{|l|}{\textbf{Class Name: Start Screen}} \\
			\hline
			\textbf{Responsibility:} & \textbf{Collaborators:} \\
			\hline
				\begin{itemize}
					\item Provide an interface that the user can interact with
					\item Send information to the GUI Controller
				\end{itemize} & 
				\begin{itemize}
					\item GUI Controller
				\end{itemize} 
				\\
			\hline
		\caption{CRC for Start Screen}
	\end{longtable}
	
	%CRC for Settings Screen
	\begin{longtable}{| p{.70\textwidth} | p{.30\textwidth} |}
			\hline
			 \multicolumn{2}{|l|}{\textbf{Class Name: Settings Screen}} \\
			\hline
			\textbf{Responsibility:} & \textbf{Collaborators:} \\
			\hline
				\begin{itemize}
					\item Display a variety of settings to the user
					\item Allow the user to verify the changes to the current settings
					\item Allow the user to go back to previous screens
				\end{itemize} & 
				\begin{itemize}
					\item GUI Controller
				\end{itemize} 
				\\
			\hline
		\caption{CRC for Settings Screen}
	\end{longtable}
	
	%CRC for Question Screen
	\begin{longtable}{| p{.70\textwidth} | p{.30\textwidth} |}
			\hline
			 \multicolumn{2}{|l|}{\textbf{Class Name: Question Screen}} \\
			\hline
			\textbf{Responsibility:} & \textbf{Collaborators:} \\
			\hline
				\begin{itemize}
					\item Receives information from the GUI Controller
					\item Takes user input from the users response for the question
					\item Sends the answer of the question to the GUI Controller
					\item Allows user to quit the current game
				\end{itemize} & 
				\begin{itemize}
					\item GUI Controller
				\end{itemize} 
				\\
			\hline
		\caption{CRC for Question Screen}
	\end{longtable}	
	
	%CRC for Map Screen
	\begin{longtable}{| p{.70\textwidth} | p{.30\textwidth} |}
			\hline
			 \multicolumn{2}{|l|}{\textbf{Class Name: Map Screen}} \\
			\hline
			\textbf{Responsibility:} & \textbf{Collaborators:} \\
			\hline
				\begin{itemize}
					\item Informs the user of the guesses on a map
					\item Informs the user of the guesses as an address
					\item Receives and passes information from the GUI Controller
					\item Requests the user feedback on the correctness of the guess
				\end{itemize} & 
				\begin{itemize}
					\item GUI Controller
				\end{itemize} 
				\\
			\hline
		\caption{CRC for Map Screen}
	\end{longtable}
	
	%CRC for Location Data
	\begin{longtable}{| p{.70\textwidth} | p{.30\textwidth} |}
			\hline
			 \multicolumn{2}{|l|}{\textbf{Class Name: Location Data}} \\
			\hline
			\textbf{Responsibility:} & \textbf{Collaborators:} \\
			\hline
				\begin{itemize}
					\item Holds the question that have been asked and their associated answers
					\item Holds the number of questions the expert himself has asked
					\item Holds most viable current guesses
					\item Provides a probability in which the guess is correct
				\end{itemize} & 
				\begin{itemize}
					\item Question Controller
				\end{itemize} 
				\\
			\hline
		\caption{CRC for Location Data}
	\end{longtable}
	
	%CRC for Environment Data
	\begin{longtable}{| p{.70\textwidth} | p{.30\textwidth} |}
			\hline
			 \multicolumn{2}{|l|}{\textbf{Class Name: Environment Data}} \\
			\hline
			\textbf{Responsibility:} & \textbf{Collaborators:} \\
			\hline
				\begin{itemize}
					\item Holds the question that have been asked and their associated answers
					\item Holds the number of questions the expert himself has asked
					\item Holds most viable current guesses
					\item Provides a probability in which the guess is correct
				\end{itemize} & 
				\begin{itemize}
					\item Question Controller
				\end{itemize} 
				\\
			\hline
		\caption{CRC for Environment Data}
	\end{longtable}
	
\newpage
	%CRC for Establishment Data
	\begin{longtable}{| p{.70\textwidth} | p{.30\textwidth} |}
			\hline
			 \multicolumn{2}{|l|}{\textbf{Class Name: Establishment Data}} \\
			\hline
			\textbf{Responsibility:} & \textbf{Collaborators:} \\
			\hline
				\begin{itemize}
					\item Holds the question that have been asked and their associated answers
					\item Holds the number of questions the expert himself has asked
					\item Holds most viable current guesses
					\item Provides a probability in which the guess is correct
				\end{itemize} & 
				\begin{itemize}
					\item Question Controller
				\end{itemize} 
				\\
			\hline
		\caption{CRC for Establishment Data}
	\end{longtable}
	
	%CRC for Location Database
	\begin{longtable}{| p{.70\textwidth} | p{.30\textwidth} |}
			\hline
			 \multicolumn{2}{|l|}{\textbf{Class Name: Location Database}} \\
			\hline
			\textbf{Responsibility:} & \textbf{Collaborators:} \\
			\hline
				\begin{itemize}
					\item  Allows for the addition of an location
					\item Allows for the removal of an location
					\item Holds a set of questions that correspond to certain location
					\item Search for location based on question results
				\end{itemize} & 
				\begin{itemize}
					\item Expert Controller 
				\end{itemize} 
				\\
			\hline
		\caption{CRC for Location Database}
	\end{longtable}
	
	%CRC for Environment Database
	\begin{longtable}{| p{.70\textwidth} | p{.30\textwidth} |}
			\hline
			 \multicolumn{2}{|l|}{\textbf{Class Name: Environment Database}} \\
			\hline
			\textbf{Responsibility:} & \textbf{Collaborators:} \\
			\hline
				\begin{itemize}
					\item  Allows for the addition of an environment
					\item Allows for the removal of an environment
					\item Holds a set of questions that correspond to certain environment
					\item Search for environment based on question results
				\end{itemize} & 
				\begin{itemize}
					\item Expert Controller
				\end{itemize} 
				\\
			\hline
		\caption{CRC for Environment Database}
	\end{longtable}
	
\newpage
	%CRC for Establishment Database
	\begin{longtable}{| p{.70\textwidth} | p{.30\textwidth} |}
			\hline
			 \multicolumn{2}{|l|}{\textbf{Class Name: Establishment Database}} \\
			\hline
			\textbf{Responsibility:} & \textbf{Collaborators:} \\
			\hline
				\begin{itemize}
					\item Allows for the addition of an establishment
					\item Allows for the removal of an establishment
					\item Holds a set of questions that correspond to certain establishments
					\item Search for establishment based on question results
				\end{itemize} & 
				\begin{itemize}
					\item Expert Controller
				\end{itemize} 
				\\
			\hline
		\caption{CRC for Establishment Database}
	\end{longtable}
	
	
	
% End Section
\newpage

\appendix
\section{Division of Labour}
\label{sec:division_of_labour}
% Begin Section
\begin{longtable}{| p{.25\textwidth} | p{.70\textwidth} |}
			\hline
			\textbf {Team Member} & \textbf{Contributions}\\ 
			\hline
			Gabriel Lopez de Leon &  Added Purpose (Section 1.1), created use case diagram for BE5 and assisted in writing the CRC.
			\\
			\hline
			Maxwell Moore & Wrote introduction(1.0), helped design BE's entered 1,3 into draw.io for submission.
			\\
			\hline
			Curtis Milo & Generated the use case diagrams for every business event and the scenarios that correspond to them. Created and refined the class analysis diagram. Created and refined the CRC cards, helped input the class responsibility collaboration cards into the document. Finally edited the grammar of the document and ensured that everything was correct according to the design ideas proposed in the team meetings.
			\\ 
			\hline
			Alexandra Rahman & Wrote the system description and helped come up with the scenarios and use cases. Created the use case diagram for BE4. Collaborated on the CRC cards and the analysis class diagram. Refined the CRC cards and the analysis class diagram. Helped input the class responsibility collaboration cards into the document. Finally, helped edit and format the document.
			\\
			\hline
			Connor Sheehan & Created use case diagram for BE3. Added overview section. Added styling.\\
			\hline
			
			\caption{Division of Labour}
		\end{longtable}



\noindent\begin{tabular}{ll}\\
	\makebox[2.5in]{\hrulefill} & \makebox[2.5in]			{\hrulefill}\\
	Gabriel Lopez de Leon & Date\\[8ex]% adds space between the two sets of signatures
	\makebox[2.5in]{\hrulefill} & \makebox[2.5in]			{\hrulefill}\\
	Curtis Milo & Date\\[8ex]
	\makebox[2.5in]{\hrulefill} & \makebox[2.5in]			{\hrulefill}\\
	Maxwell Moore & Date\\[8ex]
	\makebox[2.5in]{\hrulefill} & \makebox[2.5in]			{\hrulefill}\\
	Alexandra Rahman & Date\\[8ex]
	\makebox[2.5in]{\hrulefill} & \makebox[2.5in]			{\hrulefill}\\
	Connor Sheehan & Date\\
	\end{tabular}

% End Section


% Important notes - commented out
\iffalse
\newpage
\section*{IMPORTANT NOTES}
\begin{itemize}
%	\item You do \underline{NOT} need to provide a text explanation of each diagram; the diagram should speak for itself
	\item Please document any non-standard notations that you may have used
	\begin{itemize}
		\item \emph{Rule of Thumb}: if you feel there is any doubt surrounding the meaning of your notations, document them
	\end{itemize}
	\item Some diagrams may be difficult to fit into one page
	\begin{itemize}
		\item It is OK if the text is small but please ensure that it is readable when printed
		\item If you need to break a diagram onto multiple pages, please adopt a system of doing so and thoroughly explain how it can be reconnected from one page to the next; if you are unsure about this, please ask about it
	\end{itemize}
	\item Please submit the latest version of Deliverable 1 with Deliverable 2
	\begin{itemize}
		\item It does not have to be a freshly printed version; the latest marked version is OK
	\end{itemize}
	\item If you do \underline{NOT} have a Division of Labour sheet, your deliverable will \underline{NOT} be marked
\end{itemize}
\fi

\end{document}
%------------------------------------------------------------------------------