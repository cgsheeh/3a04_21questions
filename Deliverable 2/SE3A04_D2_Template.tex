\documentclass[titlepage]{article}

% Imported Packages
%------------------------------------------------------------------------------
\usepackage{amssymb}
\usepackage{amstext}
\usepackage{amsthm}
\usepackage{amsmath}
\usepackage{enumerate}
\usepackage{fancyhdr}
\usepackage[margin=1in]{geometry}
\usepackage{graphicx}
%\usepackage{extrarrows}
\usepackage{setspace}
\usepackage{longtable}
\usepackage{float}
\floatplacement{figure}{H}
%------------------------------------------------------------------------------

% Header and Footer
%------------------------------------------------------------------------------
\pagestyle{fancy}
\lhead{Group 2}
\chead{SFWR 3A04}
\rhead{High-Level Architectural Design} 
\renewcommand\headrulewidth{0.4pt}                                      
\renewcommand\footrulewidth{0.4pt}                                    
%------------------------------------------------------------------------------

% Title Details
%------------------------------------------------------------------------------
\title{\Huge{\textbf{High-Level Architectural Design\\21 Questions}}\vspace{9cm}}
\author{\textit{Gabriel Lopez de Leon} - lopezdg - 1310514\\\textit{Curtis Milo} - milocj - 1305877\\\textit{Max Moore} - moorem8 - 1320009\\\textit{Alex Rahman} - rahmaa25 - 1305735\\\textit{Connor Sheehan} - sheehacg - 1330964}
\date{March 7\textsuperscript{th}, 2016}                               
%------------------------------------------------------------------------------

% Document
%------------------------------------------------------------------------------
\begin{document}
\maketitle	

\tableofcontents
\newpage

\section{Introduction}
\label{sec:introduction}
% Begin Section

This section should provide an brief overview of the entire document.

\subsection{Purpose}
\label{sub:purpose}
% Begin SubSection
\begin{enumerate}[a)]
	\item Delineate the purpose of the document
	\item Specify the intended audience for the document
\end{enumerate}
% End SubSection

\subsection{System Description}
\label{sub:system_description}
% Begin SubSection
21 Questions is an android application that can be used as a location identifier whose intended use is for any user above the age 10. The application requires minimal training, experience or technical expertise to use, and can be easily picked up and used by anyone. 21 Questions is a simple games that asks the user a series of twenty-one polar or binary questions to try to identify their area of interest. In this game the area of interest is limited to an establishment, building, place, or effigy with a focus on locations only with an end goal of displaying the result through Google Maps.
% End SubSection

\subsection{Overview}
\label{sub:overview}
% Begin SubSection
\iffalse
\begin{enumerate}[a)]
	\item Describe what the rest of the document contains 
	\item Explain how the document is organised
\end{enumerate}
\fi
This document will outline the design of the \textit{21 Questions} application from an architectural perspective. The document will begin from a use case outlook, outlining application functionality from a practical point of view and taking different actors and stakeholders into consideration. Next, an analysis class diagram and associated interpretation details is outlined, to specify application behaviours and resources in a modularized form. Following this section is a detailed architectural design as well as a set of class responsibility collaboration cards. These sections specify modules in greater detail, including interfaces to be implemented in the future. The order of these sections reflects a systematic progression from requirements to a more easily constructed application.
% End SubSection

% End Section

\section{Use Case Diagram}
\label{sec:use_case_diagram}
% Begin Section
\begin{enumerate}[a)]
	\item User wants to enter new search (figure \ref{usecase:search}).
	
	\begin{center}
		\begin{figure}[H]
			\includegraphics[scale=0.6]{BE1}
			\caption{BE1}\label{usecase:search}
		\end{figure}
	\end{center}

\item An unlisted establishment requests to be included in the application (figure \ref{usecase:new_esta}).
	
	\begin{center}
		\begin{figure}[H]
			\includegraphics[scale=0.6]{BE2}
			\caption{BE2}\label{usecase:new_esta}
		\end{figure}
	\end{center}

\item Updates or maintenance of the application is required (figure \ref{usecase:updates}).
	
	%\begin{center}
		\begin{figure}[H]
			\includegraphics[scale=0.6]{BE3}
			\caption{BE3}\label{usecase:updates}
		\end{figure}
	%\end{center}

\item Management requests implementation or change of experts (figure \ref{usecase:change_expert}).
	
	\begin{center}
		\begin{figure}[h!]
			\includegraphics[scale=0.6]{BE4}
			\caption{BE4}\label{usecase:change_expert}
		\end{figure}
	\end{center}

\item User flags an incorrect or inappropriate search or result (figure \ref{usecase:badsearch}).
	
	\begin{center}
		\begin{figure}[H]
			\includegraphics[scale=0.6]{BE5}
			\caption{BE5}\label{usecase:badsearch}
		\end{figure}
	\end{center}
	
	\end{enumerate}
% End Section

\section{Analysis Class Diagram}
\label{sec:analysis_class_diagram}
% Begin Section
This section should provide an analysis class diagram for your application.
% End Section


\section{Architectural Design}
\label{sec:architectural_design}
% Begin Section
This section should provide an overview of the overall architectural design of your application. You overall architecture should show the division of the system into subsystems with high cohesion and low coupling.

\subsection{System Architecture}
\label{sub:system_architecture}
% Begin SubSection
	\begin{center}
		\begin{figure}[h!]
			\includegraphics[scale=0.6]{analysis_class_diagram}
			\caption{The analysis class diagram}\label{diagram:analysisclass}
		\end{figure}
	\end{center}

Figure \ref{diagram:analysisclass} contains the analysis class diagram. The diagram is designed to have minimal communications between the the 3 separated units. The system design is a client server model, with the client side utilizing the model-view-controller architectural pattern. The system will have two controllers, one for handling the pages of the system and the other for controlling which expert can ask questions. The information of what questions to ask is communicated from the server and stored in the clients side in an ADT. The client server style of architecture gives a very useful way of ensuring that all users will have the same information available to them without having a large amount of mobile updates. The model view controller is useful for being able to organize the structure of the clients side.


\subsection{Subsystems}
\label{sub:subsystems}
% Begin SubSection
\begin{enumerate}[a)]
	\item Provide a brief description of each subsystem. Be sure to document its purpose and relationship to other subsystems.
\end{enumerate}
% End SubSection

% End Section
	
\section{Class Responsibility Collaboration (CRC) Cards}
\label{sec:class_responsibility_collaboration_crc_cards}
% Begin Section
This section should contain all of your CRC cards.

\begin{enumerate}[a)]
	\item Provide a CRC Card for each identified class
	\item Please use the format outlined in tutorial, i.e., 
	\begin{table}[ht]
		\centering
		\begin{tabular}{|p{5cm}|p{5cm}|}
		\hline 
		 \multicolumn{2}{|l|}{\textbf{Class Name:}} \\
		\hline
		\textbf{Responsibility:} & \textbf{Collaborators:} \\
		\hline
		\vspace{1in} & \\
		\hline
		\end{tabular}
	\end{table}
	
\end{enumerate}
% End Section
\newpage

\appendix
\section{Division of Labour}
\label{sec:division_of_labour}
% Begin Section
\begin{longtable}{| p{.30\textwidth} | p{.70\textwidth} |}
			\hline
			\textbf {Team Member} & \textbf{Contributions}\\ 
			\hline
			Gabriel Lopez de Leon &  %Gabriel contributions here
			\\
			\hline
			Maxwell Moore & %Max contritubtions here
			\\
			\hline
			Curtis Milo & %Curtis contributions here
			\\ 
			\hline
			Alexandra Rahman & % Alex contritubions here
			\\
			\hline
			Connor Sheehan & Created use case diagram for BE3. Added overview section. Added styling.\\
			\hline
			
			\caption{Division of Labour}
		\end{longtable}



\noindent\begin{tabular}{ll}\\
	\makebox[2.5in]{\hrulefill} & \makebox[2.5in]			{\hrulefill}\\
	Gabriel Lopez de Leon & Date\\[8ex]% adds space between the two sets of signatures
	\makebox[2.5in]{\hrulefill} & \makebox[2.5in]			{\hrulefill}\\
	Curtis Milo & Date\\[8ex]
	\makebox[2.5in]{\hrulefill} & \makebox[2.5in]			{\hrulefill}\\
	Max Moore & Date\\[8ex]
	\makebox[2.5in]{\hrulefill} & \makebox[2.5in]			{\hrulefill}\\
	Alex Rahman & Date\\[8ex]
	\makebox[2.5in]{\hrulefill} & \makebox[2.5in]			{\hrulefill}\\
	Connor Sheehan & Date\\
	\end{tabular}

% End Section


% Important notes - commented out
\iffalse
\newpage
\section*{IMPORTANT NOTES}
\begin{itemize}
%	\item You do \underline{NOT} need to provide a text explanation of each diagram; the diagram should speak for itself
	\item Please document any non-standard notations that you may have used
	\begin{itemize}
		\item \emph{Rule of Thumb}: if you feel there is any doubt surrounding the meaning of your notations, document them
	\end{itemize}
	\item Some diagrams may be difficult to fit into one page
	\begin{itemize}
		\item It is OK if the text is small but please ensure that it is readable when printed
		\item If you need to break a diagram onto multiple pages, please adopt a system of doing so and thoroughly explain how it can be reconnected from one page to the next; if you are unsure about this, please ask about it
	\end{itemize}
	\item Please submit the latest version of Deliverable 1 with Deliverable 2
	\begin{itemize}
		\item It does not have to be a freshly printed version; the latest marked version is OK
	\end{itemize}
	\item If you do \underline{NOT} have a Division of Labour sheet, your deliverable will \underline{NOT} be marked
\end{itemize}
\fi

\end{document}
%------------------------------------------------------------------------------