\documentclass[]{article}

% Imported Packages
%------------------------------------------------------------------------------
\usepackage{amssymb}
\usepackage{amstext}
\usepackage{amsthm}
\usepackage{amsmath}
\usepackage{enumerate}
\usepackage{fancyhdr}
\usepackage[margin=1in]{geometry}
\usepackage{graphicx}
\usepackage{extarrows}
\usepackage{setspace}
%------------------------------------------------------------------------------

% Header and Footer
%------------------------------------------------------------------------------
\pagestyle{plain}  
\renewcommand\headrulewidth{0.4pt}                                      
\renewcommand\footrulewidth{0.4pt}                                    
%------------------------------------------------------------------------------

% Title Details
%------------------------------------------------------------------------------
\title{Deliverable \#2 Template}
\author{SE 3A04: Software Design II -- Large System Design}
\date{}                               
%------------------------------------------------------------------------------

% Document
%------------------------------------------------------------------------------
\begin{document}

\maketitle	

\section{Introduction}
\label{sec:introduction}
% Begin Section

This section should provide an brief overview of the entire document.

\subsection{Purpose}
\label{sub:purpose}
% Begin SubSection
\begin{enumerate}[a)]
	\item Delineate the purpose of the document
	\item Specify the intended audience for the document
\end{enumerate}
% End SubSection

\subsection{System Description}
\label{sub:system_description}
% Begin SubSection
\begin{enumerate}[a)]
	\item Give a brief description of the system. This could be a paragraph or two to give some context to this document.
\end{enumerate}
% End SubSection

\subsection{Overview}
\label{sub:overview}
% Begin SubSection
\begin{enumerate}[a)]
	\item Describe what the rest of the document contains 
	\item Explain how the document is organised
\end{enumerate}
% End SubSection

% End Section

\section{Use Case Diagram}
\label{sec:use_case_diagram}
% Begin Section
This section should provide a use case diagram for your application. 

% Business Event 1
\subsection{BE1: User starts application.}
\label{sub: BE1: User starts application.}
\begin{enumerate}[{BE1}.1]
		\item	The user presses the start button on the start screen.
		\item The system will respond by bringing them to the question screen and asking them a question.
		\item The user will answer yes, no or undecided to the question.
		\item The system will system will ask the user another question.
		\item The user will answer yes, no or undecided to the question.
		\item After 21 questions, the system will respond by displaying a map to the user.
		\item The user will hit done.
		\item The system will ask if this was the location they were thinking of.
		\item The user will answer yes it was or no it was not.
\end{enumerate}
%%%% ADD DIAGRAM!!!!


% Business Event 2
\subsection{BE2: An establishment or business wants to be added to the game.}
\label{sub: BE2:  An establishment or business wants to be added to the game.}
\begin{enumerate}[{BE2}.1]
	\item A new business opens or a business opens a new location.
	\item The business or establishment contacts the company to inform them they wish to be added to the system.
	\item The IT specialists will send them a form for the business to fill out.
	\item The business owner will supply all the necessary information.
	\item The business will return the form to the company.
	\item The IT will verify that the information is valid and add it to the system.
	\item The business is added.
\end{enumerate}
%%%% ADD DIAGRAM!!!!


% Business Event 3
\subsection{BE3: Updates or maintenance of the app is required.}
\label{sub: BE3:  Updates or maintenance of the app is required.}
\begin{enumerate}[{BE3}.1]
	\item Internal management states an issue and requests an update/ maintenance.
	\item Update/ maintenance is given a priority.
	\item The IT specialist notifies the users that the system will update and be shut down for a certain period of time, if necessary.
	\item The system will disconnect.
	\item The ?problem? will be fixed.
	\item The system will go back online.
	\item The user will be notified if they need to update the application version.
\end{enumerate}
%%%% ADD DIAGRAM!!!!


% Business Event 4
\subsection{BE4: Adding/ Removing Functionality.}
\label{sub: BE4:  Adding/ Removing Functionality.}
\begin{enumerate}[{BE4}.1]
	\item  Internal management states a function needs to be changed (added/ removed).
	\item The change is given a priority.
	\item Function will be added or removed
	\item A small focus group is selected 
	\item A survey is created
	\item Update is released to focus group
	\item Survey is sent to focus group.
	\item The update will be released depending on the results of the survey.
	\item If the survey results are not favourable, the function will be under review and released again to the focus group (repeat step 5-9). Otherwise the update is released to the general public and the user is notified that they need to update the app version. 
\end{enumerate}
%%%% ADD DIAGRAM!!!!


% Business Event 5
\subsection{BE5: Inappropriate search.}
\label{sub: BE5: Inappropriate search.}
\begin{enumerate}[{BE5}.1]
	\item A business, user or internal management recognizes that the content is inappropriate. 
	\item The ?object? shall automatically be hidden from other users
	\item The object is given a priority
	\item The object is put into a priority queue.
	\item An IT will review the ?object? and make appropriate changes (remove if necessary)
	\item Notification is sent to instigator saying the object has been reviewed and wether the changes were accepted.
\end{enumerate}
%%%% ADD DIAGRAM!!!!


% End Section

\section{Analysis Class Diagram}
\label{sec:analysis_class_diagram}
% Begin Section
This section should provide an analysis class diagram for your application.
% End Section


\section{Architectural Design}
\label{sec:architectural_design}
% Begin Section
This section should provide an overview of the overall architectural design of your application. You overall architecture should show the division of the system into subsystems with high cohesion and low coupling.

\subsection{System Architecture}
\label{sub:system_architecture}
% Begin SubSection
\begin{enumerate}[a)]
	\item Identify and explain the overall architecture of your system
	\item Be sure to clearly state the name of the architecture
	\item Provide the reasoning and justification of the choice
	\item Provide a structural architecture diagram showing the relationship among the subsystems (if appropriate)
\end{enumerate}
% End SubSection

\subsection{Subsystems}
\label{sub:subsystems}
% Begin SubSection
\begin{enumerate}[a)]
	\item Provide a brief description of each subsystem. Be sure to document its purpose and relationship to other subsystems.
\end{enumerate}
% End SubSection

% End Section
	
\section{Class Responsibility Collaboration (CRC) Cards}
\label{sec:class_responsibility_collaboration_crc_cards}
% Begin Section
This section should contain all of your CRC cards.

\begin{enumerate}[a)]
	\item Provide a CRC Card for each identified class
	\item Please use the format outlined in tutorial, i.e., 
	\begin{table}[ht]
		\centering
		\begin{tabular}{|p{5cm}|p{5cm}|}
		\hline 
		 \multicolumn{2}{|l|}{\textbf{Class Name:}} \\
		\hline
		\textbf{Responsibility:} & \textbf{Collaborators:} \\
		\hline
		\vspace{1in} & \\
		\hline
		\end{tabular}
	\end{table}
	
\end{enumerate}
% End Section

\appendix
\section{Division of Labour}
\label{sec:division_of_labour}
% Begin Section
Include a Division of Labour sheet which indicates the contributions of each team member. This sheet must be signed by all team members.
% End Section

\newpage
\section*{IMPORTANT NOTES}
\begin{itemize}
%	\item You do \underline{NOT} need to provide a text explanation of each diagram; the diagram should speak for itself
	\item Please document any non-standard notations that you may have used
	\begin{itemize}
		\item \emph{Rule of Thumb}: if you feel there is any doubt surrounding the meaning of your notations, document them
	\end{itemize}
	\item Some diagrams may be difficult to fit into one page
	\begin{itemize}
		\item It is OK if the text is small but please ensure that it is readable when printed
		\item If you need to break a diagram onto multiple pages, please adopt a system of doing so and thoroughly explain how it can be reconnected from one page to the next; if you are unsure about this, please ask about it
	\end{itemize}
	\item Please submit the latest version of Deliverable 1 with Deliverable 2
	\begin{itemize}
		\item It does not have to be a freshly printed version; the latest marked version is OK
	\end{itemize}
	\item If you do \underline{NOT} have a Division of Labour sheet, your deliverable will \underline{NOT} be marked
\end{itemize}


\end{document}
%------------------------------------------------------------------------------